\chapter{Cameron-Martin Theory: A stupidly abstract approach}

The way I like to think about Cameron-Martin theory is perhaps a bit too abstract. Recall $\calB^*$ can be `embedded' into $\calL^2(\calB, \mu)$. Rigorously consider the map $j: \calB^* \to \calL^2$ (we're not using $i$ since that's used later).
\begin{equation}
	j(f) = [f]
\end{equation}
which maps $f$ to the equivalence class of $f$ under almost sure equality. We're using $\calL^2$ instead of $L^2$ to distinguish the space of equivalence classes from the space of functions. Then $j$ is a continuous map and the RKHS $\calR_{\mu}$ is $\overline{j(\calB)}$. 

We then want embed $\calR_{\mu}$ into $\calB$. Embedding $\calR_{\mu}$ into $\calB^{**}$ is not too difficult. $j$ has an adjoint map $j^*: (\calL^2)^* \to \calB^{**}$ where $j^*(\phi) = \phi \circ j$ for all $\phi \in (\calL^2)^*$. We don't know if $j^*$ is injective. For that the following proposition from functional analysis helps.
\begin{prop}[Injective of the adjoint map]
	Let $X$ and $Y$ be Banach spaces and $j:X \to Y$ be a bounded linear map. Then the following are equivalent:
	\begin{enumerate}
		\item $j^*: Y^* \to X^*$ is injective.
		\item $j(X)$ is dense in $Y$.
	\end{enumerate}
\end{prop}
This tells us that $\calL^2$ is too big a space to look at, instead we should restrict the range of $j$ from $\calL^2$ to $R_{\mu}$. Then the above proposition tells us that $j^*: \calR_{\mu}^* \to \calB^{**}$ is an injection thus is an embedding of $\calR_{\mu}^* \to \calB^{**}$. Recall by Riesz representation there is a natural bijective linear isometry $\eq : \calR_{\mu} \to \calR_{\mu}^*$ where $\eq(f) \defeq \inn{\cdot}{f}_{L^2}$. Then $j^* \circ \eq$ embeds $\calR_{\mu}$ into $\calB^{**}$. Now let's unravel the action of this embedding, for $l \in \calB^*$.
\begin{align}
	j^* \circ \eq (f)(l) = \inn{j(l)}{f}_{L^2} = \int_{\calB} l(x) f(x) \mudif = l\left(\int_{\calB} x f(x) \mudif \right).
\end{align}
The integral inside the last quantity is a well defined Bochner integral since $\norm{f}_{L^2} < \infty$ guarantees that $x f(x)$ is integrable. This tells us that $j \circ \eq(\calB^*) \subset \ev(\calB)$ thus we can embed $\calR_{\mu}$ into $\calB$ by $k \defeq \ev^{-1} \circ j^* \circ \eq$. Removing the abstraction
\begin{equation}
	k(f) = \int_{\calB} x f(x) \mudif.
\end{equation}
Note that $k \circ j$ is exactly $\hat{C}_{\mu}$.Then $k$ is a bijection onto it's image. This gives us the Cameron-Martin space.
\begin{defn}
	Let $\mu$ be a Gaussian measure on $\calB$ with RKHS $\calR_{\mu}$ and $k$ as defined in this chapter. Then the \emph{Cameron-Martin space} of $\mu$ is
	\begin{equation}
		\calH_{\mu} \defeq k(\calR_{\mu}) \subset \calB.
	\end{equation}
	This is a Hilbert space equipped with inner product $\inn{\cdot}{\cdot}_{\mu}$ inherited from the $\inn{\cdot}{\cdot}_{L^2}$ inner product $\calR_{\mu}$ by $k$, explicitly
	\begin{equation}
		\inn{h}{k}_{\mu} \defeq \inn{k^{-1}(h)}{k^{-1}(k)}_{L^2}
		= \int_{\calB} h^*(x) k^*(x) \mudif
	\end{equation}
	where $h = \int_{\calB} x h^*(x) \mudif$ and $k = \int_{\calB} x k^*(x) \mudif$.
\end{defn}
\begin{figure}
	\centering
	\begin{tikzcd}
		\calB^* \ar[r, "j"] &
		\calR_{\mu} \ar[r, "\eq"] \ar{rrd}[sloped, below]{f \mapsto \int_{\calB} x f(x) \mudif} &
		\calR_{\mu}^* \ar[r, "j^*"] &
		\ev(\calH_{\mu}) \\
						 & & &\calH_{\mu} \ar{u}[right]{\ev}
	\end{tikzcd}
	\label{fig:cm-diagram}
	\caption{Diagram of the maps involved in Cameron-Martin theory}
\end{figure}
We've drawn the sequence of maps use in \vref{fig:cm-diagram}. The downside of this approach of constructing the Cameron-Martin space is in how long it takes to set up all the structure involved. However it does have benefits. For me, it justifies why we consider the image of $\hat{C}_{\mu}$. It's because, up to isometric isomorphisms, it is the adjoint map of the embedding of $\calR_{\mu}$ into $\calL^2$. It also allows us to reuse theorems from functional analysis. The following corollaries are immediate or can be proved very quickly from the construction of $\calH_{\mu}$.

\begin{corollary}
	There exists $C \geq 0$ such that
	\begin{equation}
		\norm{h}^2 \leq C \inn{h}{h}_{\mu}, \quad \forall h \in \calH_{\mu}.
	\end{equation}
\end{corollary}
\begin{proof}
	By definition of the Cameron-Martin space and since $k$ is bounded
	\begin{align}
		\norm{h}^2 \leq \norm{k}_{\op}^2 \norm{k^{-1}(h)}_{L^2}
		\defeq \norm{k}^2 \inn{h}{h}_{\mu}.
	\end{align}
\end{proof}
\begin{corollary}
	There is a natural isometric isomorphism between $\calH_{\mu}$ and $\calR_{\mu}$.
\end{corollary}
\begin{proof}
	$k$ is the natural isometric isomorphism. $i = k^{-1}: \calH_{\mu} \to \calR_{\mu}$ is the notation used in Prof.\ Hairer's notes.
\end{proof}
\begin{corollary}
	We have the alternate characterisation
	\begin{equation}
		\inn{h}{h}_{\mu} = \sup \left\{ l(h) : C_{\mu}(h, h) \leq 1\right\}, \quad \forall h \in \calH_{\mu}.
	\end{equation}
	Moreover $\calH_{\mu} = \{x \in \calB : \inn{h}{h}_{\mu} < \infty\}$ exactly.
\end{corollary}
\begin{proof}
	Unravelling definitions gives
	\begin{align}
		\inn{h}{h}_{\mu}
		&= \norm{k^{-1}(h)}_{L^2}
		= \norm{(\eq^{-1} \circ (j^*)^{-1} \circ \ev)(h)}_{L^2} \\
		&= \norm{((j^*)^{-1} \circ \ev)(h)}_{\op, L^2}
		= \norm{\ev(h) \circ j^{-1}}_{\op, L^2} \\
		&= \sup \left\{ (\ev(h)(j^{-1}(f)) : f \in \calR_{\mu}, \norm{f}_{L^2} \leq 1\right\} \\
		&= \sup \left\{ \ev(h)(l) : l \in \calB^*, C_{\mu}(l, l) \leq 1\right\} \\
		&= \sup \left\{ l(h) : C_{\mu}(l, l) \leq 1\right\}
	\end{align}
	where the second to last line uses density of $j(\calB^*)$ in $\calR_{\mu}$.
\end{proof}

